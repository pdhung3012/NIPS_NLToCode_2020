\begin{abstract}

Literate Programming (LP) is a programming paradigm that unites natural language (NL) with code as one document, which helps the developer to explain the logic of some parts of the code using a natural language. 
However, despite having advantages for code understanding,  LP hasn't been applied in practice mainly because there is no tool to automatically generate a correct and compilable code, and developers are required to manually specify the code for the natural language description which is a tedious and error-prone process. 
In this paper, we provide a context-aware tool called InvocMap to realize LP that allows developers for writing the description of methods inside the code environment and get the correct compilable Method Invocation (MI) code. 
Different from existing related works and other code suggestion tools such as AnyCode which only considered the given natural language description as input, we analyze the input from two different sources, i.e. NL description and its surrounding context. 
InvocMap proposes a semi-supervised approach based on three major modules. 
First, it uses unsupervised Neural Embedding to handle NL description of MIs to a list of possible method names. 
Contrary to expensive supervised machine learning methods that require a big dataset of the parallel corpus with NL and programming language.
Second, Machine Translation models trained from large scale code corpus to learn the structure of MIs, given a list of method names and surrounding code context. 
Third, a program analysis method that assigns local variables to structure of MIs to get the final code. 
InvocMap operates in two modes of input for describing MIs for developers, from method names directly, and from natural language description. 
By evaluation on data from 1000 high-quality Java projects, we got the accuracy as up to 90\% for MI suggestion from method names, and over 60\% for MI suggestion from NL description, which outperforms the prior works and shows the potential of our approach for realizing LP.


%Literate Programming (LP) is a programming paradigm that unites natural language with code as one document, which helps the developer to explain the logic of some parts of the code in a natural language. However, despite having advantages for code understanding,  LP hasn't been extensively applied in practice mainly because of two reasons. First, developers are required to specify the code for each natural language description manually which is a tedious and error-prone process. Secondly, supervised machine learning approaches for automatically retrieving code from Natural Language is expensive since it required a big dataset of the parallel corpus with NL and Programming Language. In this paper, we provide a tool InvocMap to realize LP to allow developers for writing the textual description of Method Invocations (MIs) inside code environment and get the correct MIs. Different from other code suggestion tools as AnyCode which was only considered natural language as input, we analyze the input from 2 sources: the NL description and its surrounding context. InvocMap proposes a semi-supervised approach by 3 modules. First, it uses unsupervised Neural Embedding to handle NL description of MIs to a list of possible method names. Second, Machine Translation models are trained from large scale code corpus to learn the structure of MIs given a list of method names and surrounding code context. Third, a program analysis step is proposed to assign local variables to structure of MIs to get the final code. InvocMap provides 2 modes of input for describing MIs for developers: from method names directly and from natural language description. By evaluation with training data from 1000 Java high-quality projects, we got the accuracy as up to 90\% for MI suggestion from method names and over 60\% for MI suggestion from NL description, which is outperforming the prior work AnyCode and showing the potential of realizing LP which is specified for supporting NL descriptions of MIs.
\end{abstract}
